%% PODSTAWOWE USTAWIENIA DOKUMENTU
\documentclass[12pt, a4paper, polish]{article}
%\usepackage[a4paper, lmargin=2.5cm, rmargin=2.5cm, hmargin=2.5cm, bmargin=2.5cm]{geometry}
\usepackage[a4paper,top=2.5cm,bottom=2.5cm,left=1.5cm,right=1.5cm]{geometry}
%\geometry{verbose,lmargin=2.5cm,rmargin=2.5cm}
\usepackage{enumerate}
% Symbole matematyczne
\usepackage{latexsym}
% Formatowanie czcionki
\usepackage[T1]{fontenc}
% Formatowanie polskich znaków
\usepackage{polski}
\usepackage[utf8]{inputenc}
% Akapit po sekcji
%\usepackage{indentfirst}
% Ustawienie nagłówków i stopek
\usepackage{fancyhdr}
% Liczba stron
%\usepackage{lastpage}
% Kolumny
%\usepackage{paracol}
% Czcionka latin modern
\usepackage{lmodern}
% Równania matematyczne
\usepackage{amsmath}
\usepackage{amsfonts}
\usepackage{amssymb}
\usepackage{amsthm}
% Wstawianie grafik
\usepackage{graphicx}
%\usepackage[section]{placeins} % Ogarnięcie obrazków
\usepackage{float}
\usepackage{subcaption}
%\usepackage[outdir=./]{epstopdf}
\usepackage{grffile}

%% ODSTĘPY W WIERSZACH
\setlength{\parindent}{0.5cm}
\setlength{\parskip}{0.5cm}
\linespread{1}

% SZARE TŁO TEKSTU
\usepackage[most]{tcolorbox}
\tcbset{
	frame code={}
	center title,
	left=0pt,
	right=0pt,
	top=0pt,
	bottom=0pt,
	colback=gray!30,
	colframe=white,
	width=\dimexpr\textwidth\relax,
	enlarge left by=0mm,
	boxsep=5pt,
	arc=0pt,outer arc=0pt,
}

% HIPERŁĄCZA SPISU TREŚCI
\usepackage{hyperref}
\hypersetup{
	colorlinks,
	citecolor=black,
	filecolor=black,
	linkcolor=black,
	urlcolor=black
}

\begin{document}
	\fancyhf{}	% Usunięcie domyślnego stylu numerowania
	
	% ********************** TYTUŁ ****************************
	\noindent\rule{\columnwidth}{0.2pt}
	\noindent\textsf{\begin{Large}PUT Solar Dynamics\\\end{Large}
		Opis modułu \textbf{ECM}\\
		Szymon Kacperek, Konrad Gieregowski, Jan Węgrzynowski \\
		\rule{\columnwidth}{0.2pt}}
	
	\thispagestyle{empty}
	\pagestyle{fancy}
	\fancyhead{}
	\rhead{\thepage}
	\renewcommand{\headrulewidth}{0pt}%{}
	\setlength{\footskip}{1mm}
	
	%							*****************POCZĄTEK DOKUMENTU*****************
	%\section{FORMA}
	%\begin{itemize}
	%\item Płytka z jednym procesorem 
	%Program obejmowałby trzy stany w jednym projekcie, przełączające się w zależności od stanów na pinach – przełącznikiem.
	%
	%\item Płytka z trzema procesorami
	%Każdy procesor odpowiadałby za osobny stan samochodu. Koncept ten ma na celu wyeliminowanie błędów mogących powstać podczas pracy samochodu - np. podczas jazdy włączy się tryb parkowania w wyniku błędu (o ile w ogóle jest to możliwe - kwestia do skonsultowania z kimś kompetentnym). Problemem w tym rozwiązaniu może być czas przełączania, który może stworzyć zamieszanie w elektronice. Przełączanie następowałoby również za pomocą przełącznika.
	%\end{itemize}
	
	\section{WPROWADZENIE}
	Moduł kreowany jest jako jednostka centralna samochodu. Ma za zadanie sterować przepływem danych z sieci CAN, obsługiwać wejścia oraz wyjścia, przesyłać dane do komputera głównego.
	
	ECM zasilany będzie z dodatkowego akumulatora - z intencją włączenia 24/7.
	\section{NIEZBĘDNE FUNKCJONALNOŚCI}
	\subsection{Obsługa wejść}
	\begin{description}
		\item[Sterownik świateł] 15 wejść obejmuje wszystkie funkcjonalności samochodu - otwarte drzwi/światła i inne kwestie informacyjne:
		\begin{itemize}
			\item czujniki otwartych drzwi;
			\item świateł awaryjnych;
			\item stacyjka do włączania/wyłączania samochodu;
			\item hamulec ręczny.		
		\end{itemize}
	\end{description}
	
	\subsection{Magistrala CAN, zapis oraz przesył danych}
	Moduł umożliwiał będzie przepływ danych między obiema sieciami oraz przesyłał je do komputera głównego po SPI/CAN/USB. 
	
	\subsection{Obsługa wyjść}   
	\begin{description}
		\item[Quad High Side driver] 9 wyjść;
	\end{description}
	
	\subsection{Pomiar prądów wychodzących z driverów}
	
	
	\subsection{Możliwość awaryjnego odłączenia akumulatora}
	W przypadku kolizji bądź innych błędów moduł \textit{ECM} powinien mieć możliwość odłączenia głównej baterii. Kolizja oceniana będzie na podstawie danych z akcelerometru.
	
	\section{FUNKCJONALNOŚCI DRUGORZĘDNE}
	\subsection{Zapis danych z sieci na karcie microSD}
	Według FatFS.
	
	\subsection{Analogowe czujniki}
	Czujniki podłączone pod sieć CAN.
	
\section{PERYFERIA}
\subsection{USART1}
Do debugowania. 

\subsection{USART2}
Do wykorzystania Bluetooth Low Energy.

\subsection{SDIO FATFS + DMA}
Zapis danych z CAN z użyciem DMA. Obecny pin CARD\_DETECT. 
	
	
	
	
	%\section{ZAPIS DANYCH}
	%ECM  powinien kolekcjonować dane, robiąc przebieg informacji w samochodzie w razie awarii, żeby można było zdiagnozować przyczynę powstania błędu.
	%\begin{enumerate}
	%\item Dane z sieci CAN. ECM powinien zapisywać dane z sieci CAN na karcie microSD. W tym celu płytka winna zostać wyposażona w odpowiednie interfejsy i slot który zapewni, że karta nie wypadnie podczas jazdy.
	%\item Dane z czujników. Oczujnikowanie samochodu powinno obejmować:\\
	%• Czujnik ABS \\
	%• ESP \\
	%• Zawieszenie korekcja \\
	%• Czujnik drzwi \\
	%• Czujniki hamulca, ręcznego – Paweł Czaja?\\
	%• Odkręcanie manetek - zrobione\\
	%• Kąt nachylenia świateł – do potwierdzenia – trwają rozmowy z nadwoziem, ogarnia to Marcin Miazga\\
	%• Czujnik temperatury zewnętrznej (np. PT100 bez obudowy)\\
	%• Crash sensor (np. akcelerometr)\\
	%Wszystkie dane mogą również trafiać na kartę SD.
	%\item Dane pozostałych modułów. Autorzy pozostałych projektów powinni zadbać o zapis logów z ich urządzeń, zwłaszcza danych które nie będą wysyłane poprzez sieć CAN.
	%\end{enumerate}
	%
	%\section{STEROWANIE PROCESAMI W SAMOCHODZIE}
	%Przełączanie między stanami samochodu zachodzić będzie za pomocą kluczyka lub klasycznego przełącznika. Stan parkowania byłby stanem domyślnym, następnie istniałyby dwie możliwości: I - stan jazdy, włączany przełącznikiem, II - stan ładowania, włączany automatycznie po podłączeniu ładowarki. Po przełączeniu, można opóźnić proces o 1 ms, w celu uniknięcia występowania dłuższego stanu przejściowego pomiędzy przejściem przełącznika z pozycji na pozycję.
	%Urządzenia domyślnie uruchamiają się w trybie PRE-OPERATIONAL.
	%\subsection{Stany urządzeń }
	%Kontrola nad stanami urządzeń zostaje w mocy ECM, wysyłając odpowiednie informacje po magistrali CAN.\\
	%1. OPERATIONAL - stan normalnego działania, z zezwoleniem na pracę;\\
	%2. PRE-OPERATIONAL - stan wstępnego działania, wysłanie wstępnych danych - stan ma służyć do monitorowania urządzeń, czy nie występują błędy, które wymagałyby zaprzestania pracy urządzenia oraz poinformowania użytkownika;\\
	%3. STOPPED - zatrzymanie pracy urządzenia, pozostawiając włączone peryferium CAN;\\
	%4. RESET APPLICATION - zresetowanie procesora sterującego, włącznie z resetem komunikacji;\\
	%5. RESET COMMUNICATION - zresetowanie peryferium CAN w urządzeniu, z zachowaniem normalnych warunków pracy.\\
	%\subsection{Urządzenia}
	%W przypadku kontroli działania samochodu w każdym ze stanów warto przeanalizować, jak każdy układ elektryczny (urządzenia i czujniki), powinien się zachować, osobno. W samochodzie obecne będą układy:\\
	%a. Ładowarka\\
	%b. MPPT\\
	%Dobrze by było, gdyby MPPT załączało się samoczynnie, komunikując z BMS odnośnie ładowania we wszystkich trybach samochodu. Tu już musi się dogadać Konrad z Jarem.
	%c. Akumulatory\\
	%d. Balanser\\
	%e. Dashboard\\
	%W trybie PRE-OPERATIONAL zasilony zostaje układ z diodami sygnalizującymi błąd i wyświetlacz, na którym widoczny będzie kod błędu z ewentualnym krótkim opisem.
	%f. ECM\\
	%g. Komputer główny\\
	%h. Lampy\\
	%i. Manetki\\
	%j. Napęd – falowniki i silniki\\
	%k. Panele fotowoltaiczne\\
	%l. Pedały\\
	%m. Crash sensor\\
	%n. Czujniki temperatury\\	
	%o. Czujniki hamulców
	%
	%\section{OPIS TRYBÓW PRACY SAMOCHODU}
	%\subsection{Tryb parkowania}
	%Podczas stanu zaparkowania, wszystkie urządzenia powinny być wyłączone, poza ECM oraz BMS. Zasilane są one z Supplemental battery 12V. Winna zostać załatwiona tylko kwestia ładowania poprzez panele fotowoltaiczne. Tutaj moduł MPPT będzie kierował całą pracą – więc decyzyjność w sprawie ładowania pozostawiona zostanie jemu oraz BMS, który będzie znał stan naładowania baterii oraz kontrolował proces ładowania. 
	%W przypadku, gdy MPPT nie wyśle informacji o możliwości ładowania pojazdu, moduł powinien być w trybie uśpienia pobierając jak najmniejszą liczbę energii. Kwestie:\\
	%a. Czy podczas nocy jest sens, żeby moduł pozostawał w trybie uśpienia pobierając jakąkolwiek energię? Z jakiego źródła będzie zasilany?
	%W tym wypadku warto by pochylić się nad kwestią zasilania modułu. ECM będzie miał zegarek, pokazujący godziny i w zależności od nich może włączać lub wyłączać MPPT. Jednakże, czy nie warto zostawić tego modułu włączonego cały czas, bo prąd pobierania w trybie uśpienia będzie tak niski, że można pominąć jego wartość? Ładowanie baterii może odbyć się dopiero wyzwoleniu przerwania o rozsądnej ilości prądu generowanego przez panele. Tutaj należy odpytać BMS, czy warto ładować akumulator. Włączany jest poprzez ustawienie przełącznika w tryb parkowania.Odpowiedzi układów:
	%\begin{enumerate}[a)]
	%\item Ładowarka\\
	%W stanie parkowania ładowarka powinna być wyłączona.
	%\item MPPT\\
	%Moduły MPPT winny pracować w zależności od zezwoleń BMS. Powinny jednakże być uruchomione, najlepiej w stanie oszczędzania energii. Tutaj o kwestie trzeba dopytać Jarka
	%\item Akumulatory\\
	%W tym stanie samochodu akumulator główny powinien być odłączony. Włączony jest natomiast akumulator Supplemental Battery 12 V.
	%\item Balanser\\
	%Układ BMS powinien zostać włączony, monitorując stan baterii w razie awarii oraz mieć możliwość zarządzania głównymi stycznikami w razie potrzeby ich użycia.
	%\item Dashboard\\
	%Winien zostać wyłączony.
	%\item ECM\\
	%Moduł kabinowy powinien pracować z jak najniższym zużyciem energii, odłączając zbędne układy. Winna zostać zachowana minimalna funkcjonalność, zwłaszcza gotowość na włączenie samochodu w inny tryb oraz zachować możliwość zarządzania głównymi stycznikami w razie awarii, bądź też uderzenia przez inny obiekt podczas postoju.
	%\item Lampy\\
	%Domyślnie powinny zostać wyłączone, jednakże możliwość zapalenia wszystkich rodzajów świateł poza światłami stopu powinna zostać – zwłaszcza światła awaryjne, lecz także pozostałe w razie potrzeby.
	%\item Manetki\\
	%Powinny dalej mieć możliwość sterowania światłami – tryb oszczędny, czy zwykłe zasilanie?
	%\item Napęd – falowniki i silniki\\
	%Silniki i falowniki powinny zostać wyłączone. Chyba, że z kwestii bezpieczeństwa falowniki powinny dalej monitorować stan silników?
	%Jednakże powinna zostać zachowana ostrożność i silniki powinny być odcięte fizycznie np. za pomocą stycznika, żeby nie istniała możliwość ich przypadkowego uruchomienia.
	%\item Panele fotowoltaiczne\\
	%Panele najwidoczniej nie mają wyraźnego wpływu, chyba niczego nie są w stanie zmienić.
	%\item Pedały\\
	%
	%\item Crash sensor\\
	%Czujnik powinien być zasilany bezustannie, monitorując przyspieszenia w każdej z osi i na tej podstawie uruchamiać procedurę awaryjną, w razie potrzeby. Powinien zostać umieszczony w miejscu, gdzie samochód będzie najmniej podatny na zniszczenie, żeby nie został zniszczony w momencie uderzenia. Dobrze by było rozmieścić ich kilka sztuk, w każdym punkcie obwodu samochodu.
	%\item Czujniki temperatury	\\
	%Zdecydowanie temperatura baterii powinna być cały czas monitorowana – tu w zależności od tego, czy BMS będzie miał taką możliwość. Monitorowanie temperatury na zewnątrz i wewnątrz kabiny nie wydaje się konieczne w ogóle – można zrezygnować z tego pomysłu, chyba że znajdzie się uzasadnienie.
	%\item Czujniki hamulców\\
	%W stanie parkowania nie ma konieczności informowania światłem stopu o rozpoczęciu hamowania. Bez wspomagania, chyba układ hydrauliczny w ogóle nie będzie działać.
	%\end{enumerate}
	%
	%\subsection{Tryb jazdy}
	%Włączany jest poprzez ustawienie przełącznika w tryb jazdy.\\
	%a. Ładowarka\\
	%Odłączona. Najlepiej do takiego stopnia, żeby zużywała najmniej energii.\\
	%b. MPPT\\
	%Włączone. Koordynacja pracy z BMS odnośnie zezwolenia ładowania.\\
	%c. Akumulatory\\
	%Włączony główny akumulator. Czy akumulator Supplemental jest dalej konieczny? Tu trzeba by dbać o jego ładowanie. Co będzie odpowiedzialne za ładowanie akumulatora pomocniczego?\\
	%d. Balanser\\
	%Włączony.\\
	%e. Dashboard\\
	%Włączona pełna funkcjonalność.\\
	%f. ECM\\
	%Włączona pełna funkcjonalność.\\
	%g. Lampy\\
	%Włączona pełna funkcjonalność.\\
	%h. Manetki\\
	%Włączona pełna funkcjonalność.\\
	%i. Napęd – falowniki i silniki\\
	%Włączona pełna funkcjonalność.\\
	%j. Panele fotowoltaiczne\\
	%Włączona pełna funkcjonalność.\\
	%k. Pedały\\
	%
	%l. Crash sensor\\
	%Włączona pełna funkcjonalność. Powinny być zasilane 24/7.\\
	%m. Czujniki temperatury	\\
	%Włączona pełna funkcjonalność.\\
	%n. Czujniki hamulców\\
	%Włączona pełna funkcjonalność.
	%
	%\subsection{Tryb ładowania}
	%Tryb włączany jest automatycznie po podłączeniu ładowarki oraz wyłączany po odłączeniu ładowarki. W jaki sposób, jeszcze nie wiem – wstępnie można uczulić ECM na odczyt ramek danych od ładowarki – jeśli się pojawią, wówczas będzie wiadomo że ładowarka została zasilona 230 V. Alternatywnie można wyzwolić przerwanie na którymś z pinów ECM – skorelować to z gniazdem ładowania, jakimś czujnikiem lub po prostu wyzwoleniem stanu niskiego lub wysokiego.\\
	%a. Ładowarka\\
	%Włączona pełna funkcjonalność.\\
	%b. MPPT\\
	%Włączona pełna funkcjonalność.
	%c. Akumulatory\\
	%d. Balanser\\
	%e. Dashboard\\
	%f. ECM\\
	%g. Lampy\\
	%h. Manetki\\
	%i. Napęd – falowniki i silniki\\
	%j. Panele fotowoltaiczne\\
	%k. Pedały\\
	%l. Crash sensor\\
	%m. Czujniki temperatury	\\
	%n. Czujniki hamulców\\
	%
	%\section{OBSŁUGA BŁĘDÓW}
	%Błędy obsługiwane będą w zależności od urządzenia i rodzaju błędu. Przewidziane jest utworzenie aplikacji testującej, wysyłającej ramki każdego z funkcji standardu CANopen do urządzenia docelowego. Błędy wyświetlane będą w zależności od kategorii i trybu samochodu, na dashboard i/lub na wyświetlaczu komputera głównego, wraz z opisem.
	%Ten rozdział uzupełniony będzie dopiero po zgromadzeniu wszystkich możliwych błędów, które wystąpią w samochodzie.\\
	%a. Ładowarka\\\
	%Możliwe błędy, które sygnalizuje ładowarka za pośrednictwem ramki CAN widoczne są na Rys. 5.1.
	%
	%Rysunek 5.1 Błędy sygnalizowane w ramce danych CAN przez ładowarkę
	%Zadaniem ECM w przypadku wystąpienia tych błędów będzie ich zasygnalizowanie na dashboard. Ewentualna naprawa może nastąpić poprzez modyfikację parametrów ładowania, za pośrednictwem komputera głównego z aplikacją okienkową. W przypadku błędów Bit0, Bit3 oraz Bit4 konieczna będzie reakcja sprzętowa, serwisowanie samochodu.\\
	%b. MPPT\\
	%c. Akumulatory\\
	%d. Balanser\\
	%Błędów może być kilka: \\
	%• over voltage – coś do zbicia napięcia, rozładowanie?\\
	%• Undervoltage – wyświetlenie informacji o potrzebie ładowania? \\
	%• over current – zmniejszenie zużycia energii? \\
	%• over temperaturę – wymuszenie postoju, ewentualnie uruchomienie dodatkowego chłodzenia? \\
	%• under temperature – w jaki sposób możemy ogrzać baterie?\\
	%Każdy z nich to są już poważne errory że trzeba coś zrobić już konkretnego. Skoro jest ich 5 to jeszcze może się jakieś wymyśli ale 8 bitow wystarczy. Czyli bajt z informacjami o błędach będzie wyglądać tak 0b00000000 jeśli nie będzie żadnego błędu a tak: 0b00010000 gdy np będzie overvoltage. 
	%e. Dashboard\\
	%f. ECM\\
	%Jednym z błędów ECM może być nieudana zmiana trybów pracy samochodu.
	%g. Komputer główny\\
	%h. Lampy\\
	%Kontroler świateł monitoruje wartość prądu i będzie informował o zbyt dużym/niskim prądzie.
	%i. Manetki\\
	%j. Napęd – falowniki i silniki\\
	%Tutejsze błędy będą prawdopodobnie informować również o nadmiernych lub zbyt niskich napięciach, prądach i temperaturach. Czy coś ma się tyczyć obrotów?
	%k. Panele fotowoltaiczne\\
	%l. Pedały\\
	%m. Crash sensor\\
	%W jaki sposób przewidzieć awarię czujnika? Może być monitorowanie jego odczytów. Jeśli nie pojawią się dłużej niż 1-2s, można uznać że czujnik jest zepsuty?
	%n. Czujniki temperatury	\\
	%To samo. W jaki sposób przewidzieć awarię czujnika? Może być monitorowanie jego odczytów. Jeśli nie pojawią się dłużej niż 1-2s, można uznać że czujnik jest zepsuty?
	%o. Czujniki hamulców\\
	%To samo. W jaki sposób przewidzieć awarię czujnika? Może być monitorowanie jego odczytów. Jeśli nie pojawią się dłużej niż 1-2s, można uznać że czujnik jest zepsuty?
	%
	%\section{ALGORYTM}
	%Po włączeniu do zasilania sprawdzane są aktualne pozycje przełącznika (lub kluczyków) oraz ładowarki.
	%Każde urządzenie skonfigurowane będzie tak, aby można było modyfikować jego parametry zdalnie, włączając się w sieć CAN przez odpowiednie złącze lub przez komputer główny z odpowiednią aplikacją.
	%\subsection{Tryb jazdy}
	%a. Inicjalizacja\\
	%Wszystkie urządzenia zostają włączone w stan PRE-OPERATIONAL. Ma to na celu sprawdzenie błędów, które mogą się pojawić w momencie uruchomienia urządzeń. Zostanie dodane opóźnienie 1s, żeby wszystkie wiadomości zdążyły trafić do ECM, po czym następuje zgoda na inicjalizację stanu OPERATIONAL.
	%b. Pętla główna\\
	%Jeśli nie wystąpią żadne  błędy, następuje ustawienie stanu OPERATIONAL oraz wejście programu w pętlę główną. W jej przebiegu gromadzone są dane z sieci i zapisywane na karcie SD, przesyłane do komputera głównego za pośrednictwem SPI oraz obsługa ewentualnych błędów, które wyzwalać będą przerwanie, przerywając pętle główną do funkcji obsługującej zdarzenia.
	%c. Przerwania\\
	%\subsection{Tryb parkowania}
	%
	%\subsection{Tryb ładowania}
	%Tryb jest włączany po podpięciu ładowarki do samochodu. Ładowarka wysyła informacje co 1s z informacjami na temat ładowania – o ewentualnych błędach itp. 
	%a. Inicjalizacja\\
	%Następuje wyłączenie wszystkich urządzeń poza dashboard
	%b. Pętla główna\\
	%Zawierać będzie transfer danych do dashboard, ukazując aktualne informacje o ładowaniu baterii – pochodzące od BMS oraz ładowarki.
	%\subsection{Tryb ładowania MPPT}
	%Tryb ten może współistnieć ze wszystkimi trybami pracy samochodu, zwiększając efektywność podczas zwykłego ładowania, jak i podczas jazdy oraz parkowania. Uruchamia on transfer danych do dashboard, ukazujący że ładowanie MPPT jest aktywne oraz procent naładowania baterii.
	%a. Ciało funkcji\\
	%Włączenie (jeśli to konieczne) dashboard oraz odbiór danych od MPPT. W przypadku braku uzyskiwania informacji przez określoną ilość czasu (np. 5s), tryb zostaje wyłączony.
	%
	%\section{PYTANIA}
	%- Uzupełnienie błędów w samochodzie\\
	%- Jak zostanie załatwiona kwestia regulacji dopływu powietrza do kabiny kierowcy?\\
	%- Czy bateria może być ładowana cały czas przez moduły MPPT? Czy nie będzie to wpływać negatywnie na jej żywotność?\\
	%- Czy MPPT może pracować w trybie uśpienia? Ile energii wówczas by pobierało - pomijalnie mało?
\end{document}